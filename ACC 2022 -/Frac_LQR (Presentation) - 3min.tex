
\documentclass[
rightlogo=images/presentation/acc2020,
leftlogo=images/presentation/siue.png,
leftbottomlogo=images/presentation/adroco
]{ad-conference}

\definecolor{rightcolor}{RGB}{0,0,0}
\definecolor{leftcolor}{RGB}{102,45,145}


\title[Advanced Dynamics, Robotics, and Control Research Laboratory]
{
	Closed-form Solution for The Finite-horizon Linear-quadratic Control Problem of Linear Fractional-order Systems \\ (3 mins)
} 

\author[\tiny{Advanced Dynamics, Robotics, and Control Research Laboratory}]
{
	Arman Dabiri\inst{1} \and 
	Laya Karimi Chahrogh \inst{2} \and 
	J. A. Tenreiro Machado \inst{3}}


\institute[Southern Illinois University Edwardsville]{
	\inst{1} Department of Mechanical and Mechatronics Engineering of Southern Illinois University Edwardsville \and %
	\inst{2} Department of Computer Science of Southern Illinois University Edwardsville \and
	\inst{3} Department of Electrical Engineering of Institute of Engineering, Polytechnic Institute of Porto \and
}


\date{May 25, 2021}


\bibliography{GeneralBib_2020_09_28}




\begin{document}

%%%====================================================
\begin{frame}
\titlepage	
\end{frame}


\section{Problem Statement}
\subsection{Fractional Optimal Control Problem (FOCP)}
\begin{frame}
	\begin{block}{Fractional Optimal Control Problem (FOCP)}
		Consider the LTI-FO system
		\begin{flalign}\label{eq:fLTI}
			{_{t_0}^{C} \mathcal D^{\alpha}_t \x(t)}&=A\x(t) + B \u(t)
		\end{flalign}
		with the initial condition $ \mathbf x(t_0)=\mathbf x_0 $, where $ t\in [t_0,t_f] $, $ \mathbf x(t)=[x_1(t),\ldots,x_n(t)]^T$, $ x_i(t)\in\mathbb R $, $ A\in \mathbb R^{n\times n}$, $ B\in \mathbb R^{n\times m}$, and $ \mathbf u(t) \in \mathbb R^m$.
		
		The control input $ \mathbf u(t) $ in Eq.~(\ref{eq:fLTI}) should be obtained such that minimizes the quadratic performance index
		\begin{flalign}\label{eq:cost}
			\min\limits_{\u(t)} \; \mathcal L=\frac{1}{2} \x^T(t_f)F(t_f)\x(t_f)
			\frac{1}{2}\int_{t_0}^{t_f} \parp{\x^T(t) Q \x(t)+\u^T(t) R \u(t)}\;dt,
		\end{flalign}
		where $ \x^T(t_f)F(t_f)\x(t_f) $ is the terminal cost; $ F(t_f)\in\mathbb R^{n \times n} $, $ Q\in\mathbb R^{n \times n} $ is a symmetric positive semidefinite matrix defining the energy of the controlled output, and $ R\in\mathbb R^{m \times m} $ is a symmetric positive definite matrix defining the energy of the control input.
	\end{block}
\end{frame}


\begin{frame}
	\begin{blockr}{\circled{$ \mathcal C $} Previous problems of solving linear fractional-order control problems}
		\begin{enumerate}
			\item There was no method that can obtain the optimal solution in the closed-form
			\item The current methods require NLP programming to solve even a simple FOCP. 
			\item The current methods suffers from the linear convergence in the solution.
		\end{enumerate}
	\end{blockr}
	\pause
	\begin{blockb}{\circled{$ \mathcal P $} The proposed method}
		\begin{enumerate}
			\item The optimal solution is given explicitly and in a closed form.
			\item It does not need any NLP programming techniques.
			\item The optimal solution has spectral convergence.
		\end{enumerate}
	\end{blockb}
	
\end{frame}

\subsection{Issue of Solving FOCP}
\begin{frame}[shrink=5]
	\begin{block}{An Unsovable LQR Problem for LTI-FO Systems!!}
		Let us introduce a new cost function using the Lagrange multiplier $ \boldsymbol{\lambda}(t) $ such that $ \mathcal L^* $ vanishes along the optimum trajectory. That is
		\begin{flalign}\label{eq:J}\nonumber
			&\mathcal L^*= \frac{1}{2}\x^T(t_f)F(t_f)\x(t_f)+
			\nonumber\\
			&\int_{t_0}^{t_f} \Bigg(\frac{1}{2}\parp{\x^T(t) Q \x(t)+\u^T(t) R \u(t)}+
			\boldsymbol{\lambda}^T(t)\parp{A\x(t) + B \u(t)-{_{t_0}^{C} \mathcal D^{\alpha}_t \x(t)}}\Bigg)\;dt.
		\end{flalign}
	Using the integration by parts yields
	\begin{subequations}\label{eq:cond}
		\begin{flalign}
			& _{t}^{C} \mathcal D^{\alpha}_{t_f} \boldsymbol{\lambda}(t) = A^T\boldsymbol{\lambda}(t)+Q\x(t)  ,
			\\	
			&\u(t) =- R^{-1}B^T\boldsymbol{\lambda}(t),
			\\  
			&\x(t_f)^TF(t_f)\delta \x(t_f)=\parb{\parp{_{t_0}\mathcal{J}^{1-\alpha}_t \boldsymbol{\lambda}^T(t)}\; \delta\x(t)}_{t=t_0}^{t=t_f}.
		\end{flalign}
	\end{subequations}
	\end{block}
\end{frame}


\begin{frame}[shrink=10]
	\begin{block}{}
		Since the initial condition $ \x(t_0) $ does not change, it gives
		\begin{equation}\label{eq:cond1}
			_{t_0}\mathcal{J}^{1-\alpha}_t \boldsymbol{\lambda}^T(t_f)= \x(t_f)^TF(t_f).
		\end{equation}
	\end{block}

	\begin{blockB}{}
		One can see that solving Eqs.~(\ref{eq:cond}) and (\ref{eq:cond1}) is \textbf{laborious} and \textbf{impossible} in general. 
	\end{blockB}
	\begin{blockr}{}	
		Assuming $ \boldsymbol{\lambda}(t)=P(t)\x(t) $ does not help since the Riccati type equation is 
		\begin{flalign}\label{key}\nonumber
			&{_{t_0}^C\mathcal{D}^{\alpha}_t}P(t)\;\x(t)+\frac{(t-{t_0})^{-\alpha}}{\Gamma(1-\alpha)} P({t_0}) \parp{\x(t)-\x({t_0})}+
			\\ \nonumber
			&
			\sum_{k=\ceil{\alpha}}^{\infty}
			\begin{pmatrix}
				\alpha \\
				k
			\end{pmatrix}
			{_{t_0}^C\mathcal{D}^{k}_t}{P(t)}\;{_{t_0}\mathcal{J}^{k-\alpha}_t}{\x(t)}
			= A^TP(t)\x(t)+Q\x(t),
		\end{flalign}
		which cannot be solved in practice. In literature, the last equation is simplified by considering many restrictive assumptions to obtain the similar algebraic Riccati equation for LTI systems.
	\end{blockr}
\end{frame}

\section{}
\section{}
\section{}






\end{document}